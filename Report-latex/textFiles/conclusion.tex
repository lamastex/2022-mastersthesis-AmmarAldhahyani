\section{Conclusion}

What can not be touched, smelled, or tasted can be represented visually \cite{sancho2014approach}. With all the usage of data visualization as a tool for data cleaning, data structure exploration, trends and clusters identification, presenting results, and many other usages, it does still struggle with its static form to visualize big data in a way that could help in decision making. Data quantity and the limited number of pixels in display require multiple static visualizations to show variant perspectives of the same data. 

Interactive visualization based on the ``Overview first, zoom and filter, then details on demand" \cite{shneiderman2003eyes} design strategy by Shneiderman was a way to deal with the snag that emerges when trying to statically visualize big data. Interactive visualization with its twofold strategies, navigation strategies, and visual interaction strategies opens the door to many strategies to be used when visualizing big data. On one hand, navigation strategies such as zoom \& pan, overview \& detail, and focus \& context allow the navigation between levels of detail in the data without sacrificing the whole picture. While on the other hand, visual interaction strategies such as selecting, linking, filtering, and rearranging \& remapping are more user-centered techniques that allow access to alternative perspectives to cover what big data can offer without the need for multiple static visualizations to be displayed separately. 

Visualizing big data using some or all of the interactive visualization strategies needs a powerful tool. Nowadays, there are plenty of tools and they categorized by whether they are drag and drop types or require coding skills. Drag and drop tools such as Tableau and Microsoft Power BI are powerful visualization tools and they are capable of delivering interactive visualizations in no time. They are user-friendly and they are popular between teams and companies since they can connect to many data stores and can be integrated into all major advanced databases. The drawbacks of using such tools are the limited customization and in many cases, they can not be used as showcases for users outside the tool's environment to interact with. On the other hand, tools that required coding skills such as D3 and Vega are open source tools. Many of these tools are based on JavaScript so they can be displayed on the browser by anyone around the world. For example, D3 is one of the most powerful and customizable visualization tools that can visualize countless interactive visualizations which can be customized specifically to the user's needs but on the other hand, it requires a great amount of programming knowledge in D3 and other JavaScript libraries. Other tools like Vega and Vega-Lite are based on D3 and can be programmed with fewer lines of codes compared to using D3 directly but also with less customization. 

Interactive visualization can serve preprocessed data to demonstrate the author's insights or can also serve to visualize more general data and allow the users to obtain their observations by providing them with the right interactive strategies and tools that can help them to display multiple perspectives of the data. However, customised interactive strategies require a custom use case. We have shown the power of D3 by visualizing a specific dataset in Twitter around the BlackLivesMatter and AllLivesMatter movement. We have demonstrated various techniques to shed deeper insights into the data using custom interactive visualizations in the previous Section.