\section{Future Work}

The visualization tool which was built for this project was only built for the data used in this project. A future project would be to automate this tool to accommodate many different datasets. The tool will need to be reconstructed and other libraries will need to be added to manage the projects' variables to adapt the user inputs for each visualization. New other filters will be added as well to give users a variety of options to use when visualizing their data. 

% \\\ rz

The final version of the tool will start to work by first asking the user to upload their data and a configuration file. The configuration file will be a JSON file that the user would initially need to fill up. For example, which column has the x-axis data and what is the type of this data, which column represents the data points, which column represents their radius, which filters to include and to which data's features should be assigned, and other specifications to customize the tool to their needs. There will be a guideline on how to create the configuration file which will be a straightforward procedure and will take just a few minutes to do. When the user uploads both files, the program will read through both files and assign each data feature as described in the configuration file. It will also automatically exclude the unwanted filters to have a clean display. The popup window will be designed to display the requested information in every case by using an adaptive design to work for all scenarios.  