\\\

 Unlike static data visualisation, interactive data visualisation empowers the users and allows them to specify the format used in displaying data. 
 
 \\\
 
 
 In order to pass certain knowledge about the data to the perceiver, it is important to consider Bertin's semiotics approach. Bertin, in his Semiology of Graphics \cite{bertin1983semiology} addresses different issues related to the process of creating good visualizations. He states that the designer should understand a system of related information, be able to create mapping from data to visual representation and present them on a computer screen, and importantly provide methods to interact with the visual representation and modify their presentation. Bertin emphasize the importance of the designer's ability to verify usefulness of the representation and its interaction methods. 
 
 
 Majority of existing tools today still employs concepts described almost sixty years ago in the first edition of his Semiology of Graphics. He is the one who connects most of the graphical shapes with the type of data they could represent. For instance, points represent locations, lines express a measurable length, and etc. 
 
 \\\
 
\subsection{Big Data Visualisation}

One picture is worth a thousand words. In today's world where data is being recorded from every click in the web to personal records, petabytes of data is being generated and processed everyday. It is not enough to process and analyze those data since human brains incapable to see all patterns without the data being visually represented. As big data visualisation plays important role in decision making in various fields, it's however challenging to visualize such a huge amount of data in real time or in static form.  Big data, structured and unstructured, introduces a unique set of challenges for developing visualizations. This is due to the fact that we must take into account the speed, size, and diversity of the data. A new set of issues related to performance, operability, and degree of discrimination challenge large data visualization and analysis \cite{li2015advanced}.  

 
 
 \\\
 

 
 
 \subsection{Visualisation Techniques}
 
 \subsubsection{Line Graph}
 
 It shows the relationships between items. It works well in comparing changes over a period of time.
 
 \subsubsection{Bar Chart}
 
 Bar chart can be used to compare quantities of different categories.
 
 \\\
 
 \subsubsection{Scatter plot}
 
 It is a two-dimensional plot that can be used to show variation of two items.
 
 \\\
 
 \subsubsection{Pie Chart}
 
 Pie chart is being used to compare the parts of a whole.
 
 \subsection{Applications}
 
 \subsection{Challenges}